\section{システム解析}
	前章で確定した線形モデルについてシステム解析を行う。これにより今回用いるモデルが安定化制御可能か
	判定することができる。具体的には可制御性と可観測性を調べ、それらが存在すれば安定化制御可能であるといえる。
	以下で行う計算はすべて\MaTX{}を用いた。
	\subsection{安定性}
		システムの極(Aの固有値)を計算した結果$D$を以下に示す。
		\begin{equation}
			D=\left[
			\begin{array}{c}
				7.0+0i\\
				0+0i \\
				-6.8-0i \\
				-11+0i\\
			\end{array}
			\right]
			\label{eq:Aeig}
		\end{equation}
		(\ref{eq:Aeig})式より1行目が不安定であり、2行目が安定限界であるので、今回用いるモデルは
		不安定であるといえる。
	\subsection{可制御性}
		可制御性行列は以下のようになる。
		\begin{equation}
			N_{c} = \left[
			\begin{array}{cccc}
				C & CA & CA^{2} & CA^{3} \\
			\end{array}
			\right]
		\end{equation}
		上の行列よりランクは4になれば可制御性があるといえる。ランクを計算したところ
		\begin{equation}
			rank  = 4
		\end{equation}
		となった。よって可制御性を確認できる。\\
	\subsection{可制御性}
		可観測性行列は以下のようになる。
		\begin{equation}
			N_{o} = \left[
			\begin{array}{c}
				C\\
				CA\\
				CA^{2}\\
				CA^{3}\\
			\end{array}
			\right]
		\end{equation}
		上の行列よりランクは4になれば可観測性があるといえる。ランクを計算したところ
		\begin{equation}
			rank = 4
		\end{equation}
		となった。よって、可観測性を確認できる。\\
	\par
	以上から倒立振子系の上向き基準の線形モデルは不安定なシステムであるが、4つの状態を観測することができ、
	制御することが可能なシステムといえる。次の節では倒立振子を立たせるための制御器を設計していく。
	その際にここで計算したシステム行列Aと入力行列Bを用いる。
		
%----------------------------------------------------------------------------------
\section{状態フィードバックの設計}
	制御器を設計していく第一段階として、状態フィードバック$F$の設計を行う。
	$F$は、システムを安定化する状態フィードバック
	\[
		u = -Fx
	\]
	として求めればよい。
	また、本実験においては台車が目標値へ移動を行う場合があるので、目標値$x_{ref}$
	として以下を設計することになる。
	\begin{equation}
		u(t) = F(x_{ref} - x)
		\label{eq:Feq1}
	\end{equation}
	% TODO
	この状態フィードバックの設計法には、極配置に基づく状態フィードバック測と
	LQ最適制御に基づく状態フィードバック測の2通りがあるが今回は後者の方法を用いて設計を行う。
	\par
	さて、(\ref{eq:Feq1})式をLQ問題の解として得るために、2次形式評価関数
	\begin{equation}
		J = \int_{0}^{\infty}(x^{T}Qx+Ru^{2})dt
		\label{eq:Feq2}
	\end{equation}
	\begin{equation}
		Q = diag(q_{1}^2,q_{2}^2,q_{3}^2,q_{4}^2),\ \ R = 1
		\label{eq:Feq3}
	\end{equation}
	を考える。ただし、 $diag(\ldots)$は、対角行列を表す。これは
	\begin{equation}
		J=\int_{0}^{\infty}(q_{1}^{2}r^{2}+q_{2}^{2}\theta^{2}
		+q_{3}^{2}\dot{r}^{2}+q_{4}^{2}\dot{\theta}^{2})dt
	\end{equation}
	のように表されることから、$q_1,q_2,q_3,q_4$は台車位置$r$、振り子角度$\theta$、
	台車速度$\dot{\theta}$、振子角速度$\dot{\theta}$の間のバランスをとる重み係数である。
	\par
	(\ref{eq:Feq2})、(\ref{eq:Feq3})式を最小にする(\ref{eq:Feq1})式における$F$は、
	リカッチの方程式
	\[
		A^{T}P+PA-PBR^{-1}B^{T}P+Q = 0
	\]
	の解$P>0$を求めて
	\[
		F = R^{-1}B^{T}P
	\]
	のように与えられる。

%----------------------------------------------------------------------------------
\section{最小次元オブザーバの設計}
	第二段階として、状態を推定する状態観測器(最小次元オブザーバ)
	\begin{equation}
		
	\end{equation}

%----------------------------------------------------------------------------------
\section{コントローラの離散化}

%----------------------------------------------------------------------------------
\section{振り上げ制御及び安定化の実現}

%----------------------------------------------------------------------------------