\section{安定化制御実験}
	振子を上向きに配置したときに安定化制御が可能か実験を行う。
	安定化制御実験を行った結果を示す。
	実験としては、初め振子を下向きに設置し、台車をベルトの真ん中に配置する。
	そして、データ取得のためのプログラムを起動し始めた後、シミュレーションで得られたパラメータの組を用いて
	安定化制御が可能か検証する。
	その際に、振子を真上に配置するのではなく、幾分か角度をつけてから実験を始める。
	以下にその結果とシミュレーション結果との比較を行った図を示す。
%-----------------------------------------------------------
\section{目標値の変更実験}
	台車に目標値を与えて、その目標値に台車が移動しても安定化制御可能か実験を行う。
	なお、目標値は5秒ごとに0→0.1→0のように変更される。
	以下にその結果とシミュレーション結果との比較を行った図を示す。

%-----------------------------------------------------------
\section{振り上げ制御及び安定化実験}
	振子を真下に配置し、そこから台車の動きだけで振子を振り上げ、
	安定化制御が可能か実験を行う。
	以下にその結果とシミュレーション結果との比較を行った図を示す。

%-----------------------------------------------------------