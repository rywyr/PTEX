  倒立振子実験は学部3年生の実験の講義で選択しなかったので、今回初めてでした。
  倒立振子の安定化制御のための理論をプログラムで表現し、それが予想していた
  結果を射同じ結果を見ることができて面白かったです。また、今回はjavaだけでなく
  \MaTX{}、JAMOXの3通りからのアプローチで安定化制御の実現を考えました。
  それぞれ労力ややりやすさに違いがありプログラミング言語には適材適所があることを
  改めて知ることができました。
  \par
  今回の実験で一番勉強になったのは、シミュレーションと実験は必ずしも一致しないということです。
  今まで、シミュレーションがうまくいけば実験でもうまくいくものだと決めつけていました。ですが、
  実際はそうではなく、うまくいく可能性はあるけどうまくいかない可能性もあるということでした。
  また、たとえシミュレーションでうまくいっても実験でやってみるとうまくいくが同じような結果にならない
  ということも経験しました。シミュレーションはあくまでコンピュータの中で計算した結果であり、
  現実世界の事象として発現したわけではないことを十分に理解したうえでシミュレーションを
  使っていきたいと思います。
  
  