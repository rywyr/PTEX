\section{非線形モデル}
	倒立振子の非線形モデルのコードを載せる。下向きを基準としている。
	\begin{itembox}[l]{diff\_eqs}\baselineskip=1ex
		\begin{verbatimtab}[4]
Func Matrix diff_eqs(t,x,u)
Real t;
Matrix x,u;
{
	Real r,th,dr,dth;
	Matrix xp,dxp;
	Matrix K,KZ;

	r=x(1,1);
	th=x(2,1);
	dr=x(3,1);
	dth=x(4,1);

	K=[[M + m , m *l*cos(th)]
	   [m*l*cos(th) , J + m*l*l]];

	KZ=[[-f*dr+ m*l*sin(th)*dth*dth + a * u(1,1)]
		[ m*g*l*sin(th) - c*dth ]];
	
	dxp=[[x(3:4,1)][K\KZ]]; // 倒立振子の状態の微分(非線形モデル)

	cnt = cnt+1;
	XX(1,cnt) = dxp(1,1)-Xh(1,1);
	
	// 出力を更新
	return dxp;//Xは6行の行列になる
}
		\end{verbatimtab}
	\end{itembox}
%-------------------------------------------------------------------
\section{線形モデル}
	倒立振子の線形モデルのコードを載せる。下向きを基準としている。
	\begin{itembox}[l]{main.}\baselineskip=1ex
		\begin{verbatimtab}[4]

		\end{verbatimtab}
	\end{itembox}
%-------------------------------------------------------------------
\section{システム解析}
	第4章4.1節のシステム解析に用いたコードを載せる。
	\begin{itembox}[l]{main.c}\baselineskip=1ex
		\begin{verbatimtab}[4]

		\end{verbatimtab}
	\end{itembox}
%-------------------------------------------------------------------
\section{状態フィードバック}
	状態フィードバックを設計算するのに用いたコードを載せる。ただし、LQ最適制御に基づくフィードバック則とする。
	\begin{itembox}[l]{lqr}\baselineskip=1ex
		\begin{verbatimtab}[4]
Matrix A, B, Q, R, F, P;

Q = diag(1, 1, 1, 1);
R = [1];
{F,P} = lqr(A, B, Q, R);
		\end{verbatimtab}
	\end{itembox}
%-------------------------------------------------------------------
\section{最小次元オブザーバ}
	最小次元オブザーバをゴピナスの方法で計算するのに用いたコードを載せる。
	\begin{itembox}[l]{obsg}\baselineskip=1ex
		\begin{verbatimtab}[4]
CoMatrix obs_p;

obs_p = trans([(-2,0), (-2,0)]);
{Ah, Bh, Ch, Dh, Jh} = obsg(A, B, C, obs_p);
		\end{verbatimtab}
	\end{itembox}
%-------------------------------------------------------------------	
\section{コントローラの離散化}
	連続時間オブザーバを離散化するたために用いたコードを載せる。
	\begin{itembox}[l]{c2d}\baselineskip=1ex
		\begin{verbatimtab}[4]
// ここは関数の先頭
Real dt;
Matrix Ah,Bh,Jh;
Matrix Ahd,Bhd,Jhd,Hhd;

//ここはオブザーバの設計の後
dt = 0.005;
{Ahd,Hhd} = c2d(Ah, [Bh Jh], dt);
Bhd = Hhd(:,1:2);
Jhd = Hhd(:,3);
		\end{verbatimtab}
	\end{itembox}
%-------------------------------------------------------------------
\section{シミュレーション}
	第5章でシミュレーションを行ったコードを載せる。
	\begin{itembox}[l]{InPeAboveNonLinerDesign3.mm}\baselineskip=1ex
		\begin{verbatimtab}[4]
Matrix A; //システム行列
Matrix B; //入力行列
Matrix C; //出力行列
Matrix F; //状態フィードバック行列
Matrix z; // オブザーバの状態
Matrix Ah,Bh,Ch,Dh,Jh; //オブザーバに関する行列
Matrix Ahd,Bhd,Jhd;
Matrix Xh,XX,Xtest; //推定値
Real M,m,l,J,f,a,c,g,c1,c2; //パラメータ

Func void main()
{
	Real t0,t1,r0,th0,tol;
	Real dt,dtsav; //離散化に用いる変数
	Matrix x0,z0,xp0,TC,XC,UC;
	void calcParameter();
	Matrix diff_eqs(),link_eqs();
	
	t0 = 0.0; //シミュレーション開始時刻
	t1 = 18.0; //シミュレーション終了時刻
	r0 = 0; //位置の初期値
	th0 = 10.0; //角度の初期値
	x0 = [r0, th0/180*PI ,0 ,0]'; //倒立振子の初期状態'
	z = [0,0]'; //オブザーバーの初期状態'
	dt = 0.005; //サンプリング周期
	tol=1.0E-9;//許容誤差
	dtsav = 0.05; //データ保存間隔
	
	print "Now simulating\n";
	
	cnt=0; // 初期化
    calcParameter(); //シミュレーションに必要なパラメータを計算
	// Ode()によってシミュレーションを行う
	{TC,XC,UC} =
	 Ode45HybridAuto(t0,t1,dt,x0,diff_eqs,link_eqs,tol,dtsav);
	// diff_eqs() は微分方程式を記述する関数
	// link_eqs() は複数の微分方程式の関係を記述する関数。
	// TC:時間の時系列
	// XC:状態x(t)の時系列
	// UC:入出力ベクトルの時系列
	
	//数値回を表示する。横軸t, 縦軸x(t)
	mgreplot(1,TC,XC(1,*),{"r"});
	mgreplot(2,TC,XC(2,*),{"Theta"});
	
	print TC >> "TC.mat";
	print XC >> "XC.mat";
	print UC >> "UC.mat";
}



// diff_eqs() は微分方程式を記述する関数
Func Matrix diff_eqs(t,x,u)
// tは時間
Real t;
Matrix x,u;
{
	Real r,th,dr,dth;
	Matrix xp,dxp;
	Matrix K,KZ;
	
	r=x(1,1);
	th=x(2,1);
	dr=x(3,1);
	dth=x(4,1);

	K=[[M + m , m *l*cos(th)]
	   [m*l*cos(th) , J + m*l*l]];

	KZ=[[-f*dr+ m*l*sin(th)*dth*dth + a * u(1,1)]
		[ m*g*l*sin(th) - c*dth ]];
	
	dxp=[[x(3:4,1)][K\KZ]]; // 倒立振子の状態の微分(非線形モデル)
	
	cnt = cnt+1;
	XX(1,cnt) = dxp(1,1)-Xh(1,1);

    // 出力を更新
	return dxp;//Xは6行の行列になる
}



// link_eqs() は複数の微分方程式の関係を記述する関数。
Func Matrix link_eqs(t,x)
Real t;
Matrix x;
{
    Matrix u;
    Matrix xref;
	Matrix xh,y;

    //台車の可動範囲に関する制限
    if (x(1,1) <= -0.16 || 0.16 <= x(1,1)) { // r=x(1,1)
    	OdeStop();
    }
	
	y = C*x; // 出力の計算
	xh = Ch*z + Dh*y; // 状態の推定値
	Xh= xh; //推定値を保存
	xref = [0,0,0,0]'; // 状態の目標値'
	if(0<=t && t<=5){
		xref = [0,0,0,0]'; // 状態の目標値'
	}else if(5<t&&t<=10){
		xref = [0.1,0,0,0]'; // '
	}else if(10<t&&t<=15){
		xref = [0,0,0,0]'; // '
	}
    u = F*(xref - xh);

    //入力の大きさに関する制限
    if(u(1,1) <= -15) {
        u(1,1) = -15;
    }else if (u(1,1) >= 15) {
        u(1,1) = 15;
    }

	z = Ahd*z + Bhd*y + Jhd*u; // オブザーバの状態更新
    
    //入力を更新
    return u;
}

Func void calcParameter(){

    CoMatrix pc,obs_p;
    Matrix A21,A22,B2,K,N;
	Matrix Ah ,Bh ,Jh ; // 連続時間オブザーバの係数行列
	Matrix Hhd; // 離散時間オブザーバの係数行列
	Real dt; //サンプリング周期
	Matrix Q,R,P;

    

    //物理パラメータの設定
	M=0.69;   m=0.031; l=0.15;
	J=2.5E-4; f=7.6; a=0.61;
	c=5.4E-5; g=9.8; c1=1.0;
	c2=1.0;

    // システム行列Aの準備
    K=[[M+m,m*l][m*l,J+m*l^2]];
	A21 = K~*[[0,0][0,m*g*l]];
	A22 = K~*[[-f,0][0,-c]];
	A=[[Z(2),I(2)][A21,A22]];

    //入力行列Bの準備
	B2 = K~*[[a][0]];
	B=[[Z(2,1)][B2]];

	//出力行列Cの準備
    N=[[c1,0][0,c2]];
    C=[N,Z(2)];

	//状態フィードバック行列Fを準備する
    //pc=[(-100,0),(-50,0),(-1,0),(-1,0)]'; 閉ループ系の極'
    //F = pplace(A,B,pc);
	
	//LQ最適制御のための状態フィードバック行列Fの準備
	Q = diag(1E5,1E5,1,1);
	R = [1];
	{F,P} = lqr(A, B, Q, R);

	//ゴピナスの方法で最初次元観測器を設計する
	obs_p = trans([(-30,0), (-30,0)]); // オブザーバの極
	//read obs_p; // 極を編集する
	{Ah, Bh, Ch, Dh, Jh} = obsg(A, B, C, obs_p); // ゴピナスの方法による設計

	// 連続時間オブザーバをサンプリング周期で離散化
	dt = 0.005;
	{Ahd,Hhd} = c2d(Ah,[Bh, Jh], dt); //離散化
	Bhd = Hhd(:,1:2); //係数行列の取り出し
	Jhd = Hhd(:,3); // 係数行列の取り出し
}
		\end{verbatimtab}
	\end{itembox}	
%-------------------------------------------------------------------
\section{安定化制御及び目標値変更実験}
	安定化制御及び目標値変更実験を行うために使用したコードを載せる。
	\begin{itembox}[l]{sample.mm}\baselineskip=1ex
		\begin{verbatimtab}[4]

		\end{verbatimtab}
	\end{itembox}
%-------------------------------------------------------------------
\section{振り上げ制御及び安定化実験}
	振り上げ制御及び安定化実験を行うために使用したコードを載せる。
	\begin{itembox}[l]{sample.mm}\baselineskip=1ex
		\begin{verbatimtab}[4]

		\end{verbatimtab}
	\end{itembox}
%-------------------------------------------------------------------
