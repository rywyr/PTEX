\section{理解}
16日(日曜日)と17日(月曜日)を前段階\\
18日(火曜日)を本段階\\
19日(水曜日)と20日(木曜日)を後段階\\
それ以降をデッドラインと命名する

\section{進捗}
%-------------------------------------------------------------------
	\subsection{それ以前}
		\begin{itemize}
		  \item レポートの日付を変える---後段階
		  \item TODOリストを消す---後段階
		\end{itemize}
%-------------------------------------------------------------------
	\subsection{はじめに}
		\begin{itemize}
		  \item 参考文献の[1]はどうするか---後段階
		  \item 倒立振子系の実際の写真を載せる---本段階★
		  \item 図の訂正---後段階
		  \item 文章の訂正および修正---後段階
		\end{itemize}
%-------------------------------------------------------------------
	\subsection{モデリング}
		\begin{itemize}
		  \item 図の訂正(MaTX生成の.epsファイルは位置がおかしいので要訂正)---本段階★
		  \item 文章の訂正および修正---後段階
		  \item Mとfのフィードバックによる測定の仕方を間違えているのでもう一度実験しなおしデータを再取得する---本段階★
		\end{itemize}
%-------------------------------------------------------------------
	\subsection{制御系設計}
		\begin{itemize}
		  \item 文章の訂正および修正---後段階
		  \item 図の訂正(特にする必要はないと思うが)---後段階
		\end{itemize}
%-------------------------------------------------------------------
	\subsection{シミュレーション}
		\begin{itemize}
		  \item 文章の訂正および修正---後段階
		  \item 図の訂正---後段階
		  \item 「振り上げ制御及び安定化に対する制御性能評価」を完成させる
		  \begin{itemize}
		    \item シミュレーションが間違っている可能性が高いのでJAMOXのファイルを見直す必要がある
		    \item シミュレーションの画像を取り直す→シミュレーション:考察等を書きなおす→実験:シミュレーションが絡む部分を変更
		  \end{itemize}
		\end{itemize}
%-------------------------------------------------------------------
	\subsection{実験}
		\begin{itemize}
		  \item 文章の訂正および修正---後段階
		  \item 図の訂正---後段階
		  \item 「目標値変更実験」を完成させる
		   \begin{itemize}
		     \item 12パターンの実験を行い直す(グラフが微妙なもののみ)---後段階
		     \item 図の修正(要修正)
		     \item subsectionの追加など構成を大きく変える
		  \end{itemize}
		\end{itemize}
%-------------------------------------------------------------------
	\subsection{おわりに}
		\begin{itemize}
		  \item 文章自体を考え直す---後段階
		  \item 他に書くことはないか---後段階
		\end{itemize}
%-------------------------------------------------------------------
	\subsection{参考文献}
		\begin{itemize}
		  \item 参考文系DBを完成させる---後段階
		  \item 参考文献を本文中に挿入する---後段階
		\end{itemize}
%-------------------------------------------------------------------
	\subsection{プログラム}
%-------------------------------------------------------------------
\section{図}
 \ \\
 実験を行うためのパラメータ調整\\
 ↓\\
 実験\\
 ↓\\
 実験データ取得
 ↓\\ 
 共有ディレクトリに移動\\
 ↓\\
 シミュレーションを実行(実験のパラメータで)\\
 ↓\\
 シミュレーションと実験データを重ね合わせる
 ↓\\
 図を見やすいように調整\\
 ↓\\
 保存\\
 
 \section{シミュレーションの図について}
 \begin{itemize}
   \item 線の大きさはJAMOXで設定できる
   \item 各軸のラベルもJAMOXで設定できる
   \item なんなら文字の大きさもJAMOXで設定できる
   \item 
 \end{itemize}
 
 \section{TODOリスト}
 \subsection{7月17日朝}
 今の状況はシミュレーションと実験が終わっていない状況である。\\
 シミュレーションは後「振り上げ」のみである\\
 実験は後「目標値変更」と「振り上げ」である.
 「目標値変更」は、後は実験結果を比較(シミュレーションの時の比較と同じ感じで)して考察を述べるのみ
 「振り上げ」はまだ最初の導入しか書いていないので気合いが必要かと
 \begin{itemize}
   \item ■登校
   \item ■無意味な学部ゼミ死ね
   \item ■実験:目標値変更の実験結果の図を作成(重み行列の変更、オブザーバの極変更、サンプリング周期の変更)
   \item ■実験:上記の考察を書く
   \item ■シミュレーション:振り上げのパラメータの確認決定(グラフをみて判断)
   \item ■シミュレーション:シミュレーションの図を作成してアーカイブスに
   \item ■シミュレーション:シミュレーションの際に回転してしまう場合は一回転しないように角度をいじる
   \item ■実験:シミュレーションの図を見ながらそれっぽい実験結果を記録
   \item ■シミュレーション:振り上げ制御のシミュレーション部分を完成させる
   \item ■実験:振り上げ制御の実験部分を完成させる
   \item ■ここまでで仮完成となる
   \item ■はじめに:倒立振子系の写真を載せる
   \item ■モデリング:MaTX生成のepsファイルを修正または変更
   \item ■モデリング:フィードバック制御によるMとfの求値(やり方が間違っていたので)
   \item ここまで終わってもいないのに帰ることは言語道断。認められない
   \item 泊まり込むことを覚悟してください
   \item その場合は次の日は睡眠時間や体力との兼ね合いで休むことも考慮にいれること
 \end{itemize}
 \subsection{7月17日帰宅後}
 万が一変えることができた場合\\
 たぶん17日にプレゼンの資料の作成についてお達しがあるはずなので
 そのときは帰宅してから考える。決して!大学で取り組まないように
 上記の項目が終わり次第帰宅すること
 \begin{itemize}
   \item ■はじめに:図や文章の修正
   \item モデリング:図や文章の修正
   \item 制御系設計:図や文章の修正
   \item シミュレーション:図や文章の修正
   \item 実験:図や文章の修正
   \item おわりに:図や文章の修正
   \item 参考文献:完成させる
 \end{itemize}
