\section{理解}
16日(日曜日)と17日(月曜日)を前段階\\
18日(火曜日)を本段階\\
19日(水曜日)と20日(木曜日)を後段階\\
それ以降をデッドラインと命名する

\section{進捗}
%-------------------------------------------------------------------
	\subsection{それ以前}
		\begin{itemize}
		  \item レポートの日付を変える---後段階
		  \item TODOリストを消す---後段階
		\end{itemize}
%-------------------------------------------------------------------
	\subsection{はじめに}
		\begin{itemize}
		  \item 参考文献の[1]はどうするか---本段階
		  \item 倒立振子系の実際の写真を載せる---本段階
		  \item 図の訂正---後段階
		  \item 文章の訂正および修正---後段階
		\end{itemize}
%-------------------------------------------------------------------
	\subsection{モデリング}
		\begin{itemize}
		  \item 図の訂正(MaTX生成の.epsファイルは位置がおかしいので要訂正)---後段階(できれば本段階)
		  \item 文章の訂正および修正---後段階
		  \item Mとfのフィードバックによる測定の仕方を間違えているのでもう一度実験しなおしデータを再取得する---本段階
		\end{itemize}
%-------------------------------------------------------------------
	\subsection{制御系設計}
		\begin{itemize}
		  \item 文章の訂正および修正---後段階
		  \item 図の訂正(特にする必要はないと思うが)---後段階
		\end{itemize}
%-------------------------------------------------------------------
	\subsection{シミュレーション}
		\begin{itemize}
		  \item 文章の訂正および修正---後段階
		  \item 図の訂正---後段階
		  \item 「振り上げ制御及び安定化に対する制御性能評価」を完成させる
		  \begin{itemize}
		    \item 終了:表が変なところにいく問題を解決---前段階★:文章を書いていけばおのずと詰まるよ
		    \item 一連の動作を行ってくれるプログラムの作成(前田さんの奴を参考に)(グラフの描画などを楽にさせる)---前段階★
		    \item 考察しやすいシミュレーション3パターンの画像を用意する---できるなら前段階★
		   	\item 考察をかく---本段階(できるなら前段階)★
		  \end{itemize}
		\end{itemize}
%-------------------------------------------------------------------
	\subsection{実験}
		\begin{itemize}
		  \item 文章の訂正および修正---後段階
		  \item 図の訂正---後段階
		  \item 「目標値変更実験」を完成させる
		   \begin{itemize}
		     \item 12パターンの実験を行い直す---本段階
		     \item シミュレーションの考察がお萎える図を作成---本段階
		     \item シミュレーション結果と実験結果を描画した図を作成---本段階
		     \item 結果と考察を述べる---本段階
		  \end{itemize}
		  \item 「振り上げ制御及び安定化実験」を完成させる
		  	\begin{itemize}
		  	  \item 3パターンの実験を行いなおす---本段階
		  	  \item シミュレーションの考察が行える図を作成---本段階
		  	  \item シミュレーション結果と実験結果を描画した図を作成---本段階
		  	  \item 結果と考察を述べる---本段階
		  	\end{itemize}
		\end{itemize}
%-------------------------------------------------------------------
	\subsection{おわりに}
		\begin{itemize}
		  \item 文章自体を考え直す---後段階
		  \item 他に書くことはないか---後段階
		\end{itemize}
%-------------------------------------------------------------------
	\subsection{参考文献}
		\begin{itemize}
		  \item 参考文系DBを完成させる---後段階
		  \item 参考文献を本文中に挿入する---後段階
		\end{itemize}
%-------------------------------------------------------------------
	\subsection{プログラム}
		\begin{itemize}
		  \item 終了:長いソースコードだと改ページされない問題を解決する---前段階★
		\end{itemize}
%-------------------------------------------------------------------
\section{図}
 \ \\
 実験を行うためのパラメータ調整\\
 ↓\\
 実験\\
 ↓\\
 実験データ取得
 ↓\\ 
 共有ディレクトリに移動\\
 ↓\\
 シミュレーションを実行(実験のパラメータで)\\
 ↓\\
 シミュレーションと実験データを重ね合わせる
 ↓\\
 図を見やすいように調整\\
 ↓\\
 保存\\
 
 \section{シミュレーションの図について}
 \begin{itemize}
   \item 線の大きさはJAMOXで設定できる
   \item 各軸のラベルもJAMOXで設定できる
   \item なんなら文字の大きさもJAMOXで設定できる
   \item 
 \end{itemize}
 
